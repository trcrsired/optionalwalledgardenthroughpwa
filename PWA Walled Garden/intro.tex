\section{Introduction}
\label{sec:intro}

For much of computing history, end users have enjoyed the freedom to install whatever software they desire. This is a fundamental right for end users as they own their devices and should be able to use them as they see fit. However, with the massive adoption of the Internet and web browsers in the 1980s and 1990s, this freedom gradually became a significant security issue. The Web is inherently unsafe, particularly for those who are not tech-savvy. For example, users might open their browsers such as as Google Chrome, use Google Search, visit a malicious website, downlomalware,e, and end up infecting their entire Windows operating system\cite{microsoftPreventMalware}. This can even affect other machines on the Internet. Even without downloading malwares, non-tech-savvy people can still fall victim to phishing\cite{microsoftPhishingTrends}, scareware, cryptojacking, and many other security threats. This issue persists today, so much so that nearly every Windows user relies on antivirus software, including Microsoft's own preinstalled Windows Defender. The demand for antivirus software led to incidents like the CrowdStrike outage in July 2024, which caused widespread blue screens of death across the globe. This disruption affected critical services, including airlines and medical equipment, forcing many to close temporarily\cite{wikipedia2024crowdstrike}.

In 2007, when Steve Jobs introduced the iPhone, the mobile phone ecosystem took a different direction. Users were no longer allowed to execute executables directly, as they could with traditional Windows, Linux, and MacOS systems. Software needed to be packaged as apps and run in app sandboxes. These types of operating systems are often referred to as \textit{Walled Garden} operating systems. It also became increasingly difficult for users to install alternative operating systems on their phones, as phone vendors locked down the bootloader\cite{melontini2025bootloader}. On iPhones, users could not even install third-party software\cite{apple2021trusted}. This trend led to a monopoly, with Google and Apple effectively becoming the duopoly of apps. Both Apple and Google charge a fee 30\% on app store sales, creating significant tension between developers and these tech giants\cite{FreeBSDfan2024, haney2023users}. In Richard Stallman's and the Free Software Foundation's words: \texttt{tivoization}\cite{GNUtivoization}.  This situation has escalated to the point that governments worldwide are filing antitrust lawsuits against them\cite{usdoj2024apple,europa2024apple,samr2025google}. For example, Apple has been declared a gatekeeper and is forced to allow sideloading in the European Union\cite{europa2024apple}. Despite this duopoly, users still face scams and other security threats\cite{10.1145/3548606.3560615, 10.1145/2484313.2484316}.

The monopolies of Google and Apple have had serious long-term consequences. With the failure of Microsoft's Windows Phone due to app gaps, Windows, as the most important platform for running critical business software, cannot implement similar restrictions. Developers do not create apps for the Microsoft Store as they do for the Google Play Store or Apple App Store. Microsoft offers Windows in S mode\cite{MicrosoftSModeFAQ}, but it lacks apps, making it less attractive compared to Android tablets or Chromebooks, which have a wider range of apps, better security, and are often much cheaper. In addition to the shortage of apps, Windows users dislike the restrictive nature of walled gardens\cite{pcgamerUniversalApps, redditRufusUWP}. This preference stems from Windows' long-standing tradition of distributing applications through third-party sources, which has proven to be highly effective. This situation also does not help other operating systems such as Linux. It also leads to planned obsolescence, contributing to environmental pollution, as many Android phones from five years ago no longer receive updates from vendors\cite{androidcentralLinuxKernel}. The outdated Linux kernel on these devices makes them vulnerable to attacks\cite{androidvulnerabilitiesByYear}.

Clearly, this is a textbook example of the contradictions between security and usability. If you prioritize absolute freedom, you end up with numerous security issues. On the other hand, if you prioritize absolute security, the device becomes practically useless. What is the point of using electronic devices like phones, PCs, or VR headsets when they are inherently less secure than not using them? This often becomes an excuse for big tech companies like Apple\cite{apple2021trusted,FederighiPrivacyKeynote2021, Dobie2022, MentalOutlawSideloading2021} and Google to justify their monopolistic practices. Over time, users have been steadily losing control over their own hardware and software rights\cite{win11stop}. The question then arises: why can't we have both security and usability\cite{nistUsersNotStupid}?

In this paper, we will explore a new security model applicable to Windows and other operating systems. Windows serves as an excellent example because it remains a significant target for malware, although its market share is gradually being replaced by Android as more people use phones over PCs. Despite this shift, we still rely heavily on our computers, so improving Windows security is crucial. Unlike phones, which are heavily locked down as previously mentioned, computers offer more flexibility for tasks such as programming.

With the rise of Progressive Web Apps (PWAs)\cite{8399228, webDevPWA}, many websites now support this technology. PWAs can perform a variety of core Apps, including music player apps like Snae Player\cite{snaeplayer} and EPUB reader apps like Flow\cite{flowossApp}. Many core on-line services also support PWAs, such as Microsoft Office, Onedrive, YouTube, TikTok, Spotify, Apple Music, Starbucks, Tinder, Instagram, Twitch, Overleaf, Battle.net, Facebook, X (Twitter), Bestbuy, Discord, Wikipedia, Reddit, LinkedIn, Uber, TradingView, Yahoo Finance, Chase, VSCode, Microsoft Copilot, ChatGPT, DeepSeek, Murlok.io, Pornhub, Tencent News, Taobao, RT, PressTV, and GitHub. Even virtual machine apps, such as virtual machines in Windows 95 using v86, support PWAs. Although many social media platforms often promote their own mobile applications and limit the functionality of their mobile PWAs, these PWAs still serve users effectively on a daily basis.

The Web browser naturally acts as a sandbox, greatly reducing security risks compared to native apps on Windows, as Windows Win32 binaries typically do not run in app sandboxes. Microsoft itself can provide a wide range of core services, including Microsoft Office, Teams, OneDrive, and Copilot, to fill that gap. Therefore, we can create an optional walled garden login toggle for non-tech-savvy people to use these devices through PWAs on Windows while preserving the versatility and openness of Windows as a platform.

\subsection{S/P Toggle}
In this paper, we propose a new login screen toggle for future Windows 12, known as the \texttt{S/P toggle}. Inspired by Microsoft's Windows S mode, this feature is no longer a separate Windows version but a login screen toggle for standard user mode on Windows Home and Professional editions. We refer to the traditional unrestricted Windows user mode as P mode, where P represents Power user mode. This toggle allows users to easily switch between simple user mode and Power user mode. With this new toggle, separate Windows S mode versions will be removed and will no longer exist. Existing devices still using Windows S mode versions will automatically be unlocked.

In simple user mode (\texttt{S mode}), users are restricted to a locked-down, Walled Garden Mode where everything is an App. In this mode, users are only allowed to install progressive web apps from the Microsoft Store and are NOT allowed to directly use any web browsers or search engines, including Google Chrome and Google Search. Even Microsoft's own Microsoft Edge browser and Bing is not directly accessible and can only be used for loading Progressive Web Apps. The Microsoft Store should provide installations for Progressive Web Apps and Chromium Web Extensions. Since Progressive Web Apps are SEO-friendly, a search engine still exists in this mode, but it will only search content within the installed Progressive Web Apps. Additionally, users can use generative AI chatbots like Microsoft Copilot, ChatGPT and DeepSeek PWAs for searching web content. The user interface should resemble those of Android and iOS devices, making it more touch-friendly.

In power user mode (\texttt{P mode}), Windows operates as it does today, allowing users to run \texttt{.exe} PE files, install third-party software or web browsers, use the Windows Subsystem for Linux, and access UEFI settings to install alternative operating systems like Linux or FreeBSD. P mode also includes features for managing S mode, such as sideloading Progressive Web Apps and Chromium Web Extensions, as well as configuring app permissions. While users can enable Microsoft Edge or install another sandboxed browser in S mode, the operating system provides warnings and encourages users to sideload websites as Progressive Web Apps instead of using web browsers directly, or remain in P mode for unrestricted functionality.

The S/P mode toggle is exclusively a login screen UI feature and does not affect other remote functionalities, such as SSH or remote desktop.

This \texttt{S/P} toggle can easily be adopted by other operating systems like Linux or macOS to enhance security, and by Android and iOS to improve usability. However, in this paper, we will primarily focus on Windows since it is well understood by most readers and its adoption is more practical in the near term.

%%% Local Variables:
%%% mode: latex
%%% TeX-master: "main"
%%% End:

%  LocalWords:  biometrics cryptographic parallelized lossy
